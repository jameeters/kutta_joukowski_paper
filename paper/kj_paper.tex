\documentclass[letterpaper, twoside, 12 pt]{article}
\usepackage{james_kj}

% Use this to toggle professionalism
\newif\ifpresstime
\presstimefalse
\presstimetrue

\ifpresstime
	\renewcommand{\meo}[1]{}
\fi

\title{The Kutta--Joukowski Theorem}
\date{\today}
\author{James McFeeters}
\begin{document}
	\maketitle
	% \newcounter{y}
	% \setcounter{y}{0}

\begin{abstract}
	The Kutta--Joukowski theorem is an elegant result that characterizes the lifting force produced by an object in a fluid flow. 
	It is applicable to the analysis to airfoils at low speeds.
	This paper will present, in full and with explanation, a derivation of the Kutta--Joukowski theorem.
	This will be prefaced by an introduction to the use of complex analysis in analyzing problems of fluid dynamics.
\end{abstract}

\setlength{\parindent}{0 pt}
\setlength{\parskip}{1 em}

\section{Flows} % (fold)
\label{sec:flows}
	We will be using techniques from complex analysis to examine fluid flow in two dimensions.
	Imagine a 3-dimensional fluid flow which is identical in some sequence of parallel planes.
	We use the complex plane as a cross-sectional slice of this flow, examining one of those planes.

	We assume that the flow in in a \textit{steady-state}: that the flow does not change over time.
	We also assume that the fluid is \textit{incompressible}: that the density of the fluid is the same throughout the plane.
	In terms of aerodynamics, this is equivalent to a restriction to subsonic speed regimes.

	\begin{definition}[Flow]
		A \textit{flow} is described by a function $f = u + iv$, which gives the velocity (speed and direction) of a particle of fluid at each point in the plane.
		If $z = x + iy$, then $u(x, y)$ gives the speed of a particle in the x-direction, or along the real axis, at point $z$, and $v(x, y)$ gives its speed in the y-direction (or along the imaginary axis).
	\end{definition}

	\paragraph*{Curves and Flows}
	Now we will introduce a framework for the discussion of curves immersed in the flow.
	Let $\gamma$ be a piecewise smooth Jordan curve of length $l$ described by $\gamma = \{ z(s) \mid z(s) = x(s) + i y(s), \ 0 \leq s \leq l \}$.
	Then the number $\tau(s) = z'(s) = \frac{dx}{ds} + i \frac{dy}{ds}$ is of unit modulus and tangent to $\gamma$ at $z(s)$.
	The number $n(s) = \frac{1}{i} z'(s) = \frac{dy}{ds} - i \frac{dx}{ds}$ is of unit modulus and normal to $\gamma$ at $z(s)$.

	Now we can describe the components of $f$ tangent and normal to $\gamma$:
	\[
		f_\tau = u \frac{dx}{ds} + v \frac{dy}{ds}
	\]
	gives the component of the flow tangent to $\gamma$ at each point along its length, while
	\[
		f_n = -v \frac{dx}{ds} + u \frac{dy}{ds}
	\]
	gives the component of the flow normal to $\gamma$.
	See Figure~\ref{fig:flow_tangent_and_normal}

	\begin{figure}[H]
		\centering
		\begin{tikzpicture}[>=stealth]
			\coordinate (z) at (1, 0.7);
			% \draw[tangent=0.5] plot [smooth, tension = 1] coordinates {(0, 0) (z) (3, -0.3)} node[below] {$\gamma$};
			\draw[tangent=0.361] (0, 0) to[out = 10, in = 180]  (z) to[out = -10, in = 160] (3, -0.3) node[below] {$\gamma$};
			\filldraw (z) circle (1 pt) node[below] {$z$};
			\coordinate (fz) at ($(z)+(1, 1)$);
			\draw[->, use tangent] (0, 0) -- (1.5,0) node [right] {$f_\tau$};
			\draw[->, use tangent] (0, 0) -- (0, 1)  node [left]  {$f_n$};
			\draw[->, thick, use tangent] (0, 0) -- (1.5, 1) node[right] {$f$};
		\end{tikzpicture}
		\captionsetup{width = 0.5 \textwidth}
		\caption{A flow $f$, with components tangent and normal to a curve $\gamma$.}
		\label{fig:flow_tangent_and_normal}
	\end{figure}

	\begin{definition}[Sourceless / Sinkless]
		We call a flow \textit{sourceless} in a domain $G$ if no fluid is created at any point in $G$, and \textit{sinkless} if no fluid is destroyed.
		(For brevity, we will use ``sourceless'' to indicate both of these properties.)

		Let $\gamma$ be a closed, piecewise smooth Jordan curve in $G$.
		Then $f$ is sourceless if 
		\[
			\int_\gamma f_n(z) \ ds = 0
		\]
		for all $\gamma$ in $G$.
	\end{definition}

	\begin{definition}[Irrotational]
		A flow is referred to as \textit{irrotational} if it contains no vortices or eddies.
		Let $\gamma$ be a closed piecewise smooth curve in $G$.
		The flow $f$ is irrotational in $G$ if 
		\[
			\int_\gamma f_\tau \ ds = 0
		\]
		for all $\gamma$ in $G$.
		That is, there is no net movement of fluid along the edge of any closed curve in $G$.
	\end{definition}


	\begin{theorem}
		Let a flow $f$ be sourceless and irrotational on a domain $G$.
		Then the conjugate $\bar f$ of this flow is analytic on $G$.
	\end{theorem}
	\begin{proof}
		Now consider the conjugate of the flow function: $\bar f = u - i v$.
		For any curve $\gamma$, 
		\begin{align*}
			\int_\gamma \bar f \ dz &= \int_\gamma (u - iv) (dx + i dy) \\
			&= \int_\gamma u\> dx + v\> dy \ + \ i \! \int_\gamma -v \> dx + u \> dy\\
			&= \int_\gamma f_\tau \ ds + i \int_\gamma f_n \ ds
		\end{align*}

		So we know that 
		\[
			\re \int_\gamma \bar f \ dz = 0 \text{ when $f$ is irrotational}
		\]
		and
		\[
			\im \int_\gamma \bar f \ dz = 0 \text{ when $f$ is sourceless}
		\]
		By applying Green's theorem, we find that 
		\begin{align*}
			\re \int_\gamma \bar f \ dz &= \iint_{\intt \gamma} \left( \frac{\partial v}{\partial x} - \frac{\partial u}{\partial y} \right) dx\ dy 
			\quad = 0 \quad \text{ when $f$ is irrotational}\\[1 ex]
			\im \int_\gamma \bar f \ dz &= \iint_{\intt \gamma} \left( \frac{\partial u}{\partial x} + \frac{\partial v}{\partial y} \right) dx\ dy 
			\quad = 0 \quad \text{ when $f$ is sourceless}
		\end{align*}

		Thus we know that on the interior of every $\gamma$, 
		\begin{align*}
			\frac{\partial u}{\partial y} - \frac{\partial v}{\partial x} &= 0 \implies \frac{\partial u}{\partial x} = \frac{\partial v}{\partial y} & &\text{when $f$ is irrotational} \\
			\text{and}\hspace{5 em}&\\
			\frac{\partial u}{\partial x} + \frac{\partial v}{\partial y} &= 0 \implies \frac{\partial u}{\partial x} = - \frac{\partial v}{\partial y} & &\text{when $f$ is sourceless} \\
		\end{align*}
		Thus $\bar f$ satisfies the Cauchy--Riemann equations in $G$.
	\end{proof}

	Since $\bar f$ is an analytic function, we know by the fundamental theorem of calculus that it must be the derivative of some analytic function.
	We will see that this function has beautiful properties.
% section flows (end)

\section{The Complex Potential} % (fold)
	\label{sec:complex_potential}

	\begin{definition}[Complex Potential]
	\label{def:complex_potential}
		Let $f$ be a sourceless and irrotational flow in a simply connected domain $G$.
		The \textit{complex potential} of the flow is the function $F$ such that:
		\[
			F'(z) = \bar f(z) \qquad \text{or} \qquad F(z) = \int_{z_0} ^z \bar f \ ds
		\]
		where $z_0$ is some fixed point in $G$.
	\end{definition}

	\begin{theorem}
		Given a sourceless and irrotational flow in a domain $G$ there exists a complex potential $F$ satisfying definition~\ref{def:complex_potential}.
	\end{theorem}
	\begin{proof}
		Let $f$ be as specified in definition~\ref{def:complex_potential}.

		We know that the integral
		\[
			\int_{z_0}^z \bar f \ ds = \int_{z_0}^z u\> dx + v\> dy \ + \ i \! \int_{z_0}^z -v \> dx + u \> dy
		\]
		is independent of path.
		Thus we know that the differential forms $u\> dx + v\> dy$ and $-v\> dx + u\> dy$ are exact.
		That is, there exist real functions $\phi$ and $\psi$ of two real variables such that 
		\begin{align*}
			d \phi &= u\> dx + v\> dy \\
			\text{and}\qquad &\\
			d \psi &= -v\> dx + u\> dy \\
		\end{align*}
		This implies the following:
		\begin{alignat*}{2}
			u &= \frac{\partial \phi}{\partial x},  \qquad	& v &= \frac{\partial \phi}{\partial y} \\[1 em]
			-v &= \frac{\partial \psi}{\partial x}, \qquad	& u &= \frac{\partial \psi}{\partial y} \\
		\end{alignat*}
		So that the functions $\phi$ and $\psi$ satisfy the Cauchy--Riemann equations.
		Thus we define 
		\[
			F = \phi + i \psi
		\]
		which implies
		\[
			F' = \frac{\partial \phi}{\partial x} + i \frac{\partial \psi}{\partial x} = u - iv 
		\]
		Thus $F$ is analytic on $G$, and satisfies the definition of the complex potential.
	\end{proof}

	\begin{remark}
		The proof demonstrates that given a flow we may find its complex potential, but it also implies the converse.
		Given a complex potential $F$ analytic on a domain $G$, we may find a sourceless and irrotational flow function on $G$.
		This fact is key to the most beautiful properties of our model of fluid motion.
	\end{remark}


% section complex_potential (end)

\section{Streamlines} % (fold)
\label{sec:streamlines}

	Before we can describe streamlines, we must introduce a simple term:\
	\begin{definition}[Level Curve]
		Let $G$ be a domain and let $q$ be a real-valued function on $G$.
		Given a constant $k$, the set 
		\[
			\{ z \in G \mid q(z) = k \}
		\]
		is called a \textit{level curve} of $q$.
	\end{definition}

	\begin{definition}(Streamlines and Equipotentials)
		Let $F = \phi + i \psi$ be a complex potential.

		The function $\phi$ is referred to as the \textit{velocity potential}, because its partial derivatives give the horizontal and vertical components of the flow function.
		The level curves 
		\[
			\{ z \in G \mid \phi(z) = \mathrm{const.} \}
		\]
		are called equipotentials, because they describe lines of constant potential velocity.

		The function $\psi$ is called the \textit{stream function} because it describes the paths followed by particles in the flow.
		The level curves 
		\[
			\{ z \in G \mid \psi(z) = \mathrm{const.} \}
		\]
		are called \textit{streamlines} of the flow $f$.
		These lines give the possible paths for a particle in the flow.
	\end{definition}

	\begin{figure}[H]
		\centering
		\begin{tikzpicture}
			\newcount\scale
			\advance\scale by 3
			
			\newcount\angle
			\advance\angle by 0

			\draw[scale = \the\scale, rotate = \the\angle, fill = lightgray] plot file{../airfoils/NACA_0030.dat} -- cycle;

			\draw [draw = none, rotate = \the\angle] (0, 0) to coordinate[pos=0] (LE) coordinate[pos=0.5] (mid)  coordinate[pos=1] (TE) (\the\scale, 0);

			\newcount\wdist
			\advance\wdist by -2

			\newdimen\wspace
			\advance\wspace by 2 pt

			\newdimen\wgap
			\advance\wgap by 10 pt

			\draw[draw = none](\the\wdist, 0) to coordinate [pos = 0] (wind_start) (LE);
			\draw[draw = none](TE) -- ($(TE) + (-\the\wdist, 0)$);

			\coordinate(top_1) at (0.175\the\scale , 0.4\the\scale);
			\coordinate(top_2) at ($(top_1) + (0.5, 0.15)$);
			\coordinate(top_3) at ($(top_2) + (1, -0.075)$);
			\coordinate(top_4) at ($(top_3) + (1, -0.23)$);
			\coordinate(top_5) at ($(top_4) + (2.1, -0.09)$);

			\draw[wind]($(wind_start)+(0, \the\wgap )$) to [out = 0, in = 190] (top_1);
			\draw[wind, shorten <= 0 pt](top_1) [out = 10, in = 190] to (top_2);
			\draw[wind, shorten <= 0 pt](top_2) [out = 10, in = 170] to (top_3);
			\draw[wind, shorten <= 0 pt](top_3) [out = -10, in = 170] to (top_4);
			\draw[wind, shorten <= 0 pt](top_4) [out = -10, in = 180] to (top_5);


			\coordinate(botm_1) at (0.175\the\scale , -0.4\the\scale);
			\coordinate(botm_2) at ($(botm_1) + (0.5, -0.15)$);
			\coordinate(botm_3) at ($(botm_2) + (1, 0.075)$);
			\coordinate(botm_4) at ($(botm_3) + (1, 0.23)$);
			\coordinate(botm_5) at ($(botm_4) + (2.1, 0.09)$);

			\draw[wind]($(wind_start)+(0, -\the\wgap )$) to [out = 0, in = 170] (botm_1);
			\draw[wind, shorten <= 0 pt](botm_1) [out = -10, in = 170] to (botm_2);
			\draw[wind, shorten <= 0 pt](botm_2) [out = -10, in = 190] to (botm_3);
			\draw[wind, shorten <= 0 pt](botm_3) [out = 10, in = 190] to (botm_4);
			\draw[wind, shorten <= 0 pt](botm_4) [out = 10, in = 180] to (botm_5);

			\coordinate(top_22) at ($(top_2) + (0, \the\wgap)$);
			\coordinate(top_24) at ($(top_4) + (0, \the\wgap)$);
			\coordinate(top_25) at ($(top_5) + (-0.2, \the\wgap)$);

			\draw[wind]                 ($(wind_start)+(0, \the\wgap + \the\wgap )$) to [out = 0, in = 190] (top_22);
			\draw[wind, shorten <= 0 pt](top_22)                                     to [out = 10, in = 170] (top_24);
			\draw[wind, shorten <= 0 pt](top_24)                                     to [out = -10, in = 180] (top_25);

			\coordinate(botm_22) at ($(botm_2) + (0, -\the\wgap)$);
			\coordinate(botm_24) at ($(botm_4) + (0, -\the\wgap)$);
			\coordinate(botm_25) at ($(botm_5) + (-0.2, -\the\wgap)$);

			\draw[wind]($(wind_start)+(0, -\the\wgap - \the\wgap )$) to [out = 0, in = 170] (botm_22);
			\draw[wind, shorten <= 0 pt](botm_22)                                     to [out = -10, in = 190] (botm_24);
			\draw[wind, shorten <= 0 pt](botm_24)                                     to [out = 10, in = 180] (botm_25);
		\end{tikzpicture}
		\captionsetup{width = 0.5 \textwidth}
		\caption{
			Streamlines around an object in a flow.
			Every particle follows a streamline, no particle ever crosses a streamline.
		}
		\label{fig:streamlines}
	\end{figure}

	Since a streamline is a curve along which $\psi$ is constant, $\psi'$ must be uniformly zero along a streamline.
	That is, along a streamline:
	\begin{align*}
		\frac{\partial \psi}{\partial x}dx + \frac{\partial \psi}{\partial y}dy &= 0 \\
		-v\> dx  + u\> dy &= 0 \\
		f_n &= 0
	\end{align*}
	Thus we find that there cannot be any flow normal to a streamline, implying that along a streamline $f = f_\tau$ the flow is always tangent to a streamline.

	By similar reasoning, we may find that the flow is always normal to an equipotential, meaning that equipotentials and streamlines are everywhere orthogonal to each other.

	Consider the object in the flow in Figure~\ref{fig:streamlines}.
	Clearly its boundary is impermeable to the fluid --- there is no flow across its boundary at any point.
	Thus the boundary of this object must be a streamline of the flow.
	This means that given the streamlines of a flow, one may find a stream function, and by its conjugate harmonic function, may complete the complex potential whose derivative will give the original flow function.

	This is a powerful fact from a computational standpoint.
	Given a model in a wind tunnel, measuring the wind speed at every point around our model would be work-intensive, and would probably disrupt the flow, leading to inaccurate results.
	It is easy, on the other hand, to release smoke into the flow, and thereby track the motion of the air and determine the streamlines by careful observation.

	


	

% section streamlines (end)

\clearpage
\nocite{*}
\bibliographystyle{plainnat}
\bibliography{kj_references}



\clearpage
\rule{\textwidth}{5 pt}
\begin{figure}[H]
	\centering
	\begin{tikzpicture}
		\newcount\scale
		\advance\scale by 3
		
		\newcount\angle
		\advance\angle by 0

		\draw[scale = \the\scale, rotate = \the\angle, fill = lightgray] plot file{../airfoils/NACA_0030.dat} -- cycle;

		\draw [draw = none, rotate = \the\angle] (0, 0) to coordinate[pos=0] (LE) coordinate[pos=0.5] (mid)  coordinate[pos=1] (TE) (\the\scale, 0);

		\newcount\wdist
		\advance\wdist by -2

		\newdimen\wspace
		\advance\wspace by 2 pt

		\newdimen\wgap
		\advance\wgap by 10 pt

		\draw[wind](\the\wdist, 0) to coordinate [pos = 0] (wind_start) (LE);
		\draw[wind](TE) -- ($(TE) + (-\the\wdist, 0)$);


		\coordinate(top_1) at (0.175\the\scale , 0.4\the\scale);
		\coordinate(top_2) at ($(top_1) + (0.5, 0.15)$);
		\coordinate(top_3) at ($(top_2) + (1, -0.075)$);
		\coordinate(top_4) at ($(top_3) + (1, -0.23)$);
		\coordinate(top_5) at ($(top_4) + (2.1, -0.09)$);

		\draw[wind]($(wind_start)+(0, \the\wgap )$) to [out = 0, in = 190] (top_1);
		\draw[wind, shorten <= 0 pt](top_1) [out = 10, in = 190] to (top_2);
		\draw[wind, shorten <= 0 pt](top_2) [out = 10, in = 170] to (top_3);
		\draw[wind, shorten <= 0 pt](top_3) [out = -10, in = 170] to (top_4);
		\draw[wind, shorten <= 0 pt](top_4) [out = -10, in = 180] to (top_5);


		\coordinate(botm_1) at (0.175\the\scale , -0.4\the\scale);
		\coordinate(botm_2) at ($(botm_1) + (0.5, -0.15)$);
		\coordinate(botm_3) at ($(botm_2) + (1, 0.075)$);
		\coordinate(botm_4) at ($(botm_3) + (1, 0.23)$);
		\coordinate(botm_5) at ($(botm_4) + (2.1, 0.09)$);

		\draw[wind]($(wind_start)+(0, -\the\wgap )$) to [out = 0, in = 170] (botm_1);
		\draw[wind, shorten <= 0 pt](botm_1) [out = -10, in = 170] to (botm_2);
		\draw[wind, shorten <= 0 pt](botm_2) [out = -10, in = 190] to (botm_3);
		\draw[wind, shorten <= 0 pt](botm_3) [out = 10, in = 190] to (botm_4);
		\draw[wind, shorten <= 0 pt](botm_4) [out = 10, in = 180] to (botm_5);

		\coordinate(top_22) at ($(top_2) + (0, \the\wgap)$);
		\coordinate(top_24) at ($(top_4) + (0, \the\wgap)$);
		\coordinate(top_25) at ($(top_5) + (-0.2, \the\wgap)$);

		\draw[wind]                 ($(wind_start)+(0, \the\wgap + \the\wgap )$) to [out = 0, in = 190] (top_22);
		\draw[wind, shorten <= 0 pt](top_22)                                     to [out = 10, in = 170] (top_24);
		\draw[wind, shorten <= 0 pt](top_24)                                     to [out = -10, in = 180] (top_25);

		\coordinate(botm_22) at ($(botm_2) + (0, -\the\wgap)$);
		\coordinate(botm_24) at ($(botm_4) + (0, -\the\wgap)$);
		\coordinate(botm_25) at ($(botm_5) + (-0.2, -\the\wgap)$);

		\draw[wind]($(wind_start)+(0, -\the\wgap - \the\wgap )$) to [out = 0, in = 170] (botm_22);
		\draw[wind, shorten <= 0 pt](botm_22)                                     to [out = -10, in = 190] (botm_24);
		\draw[wind, shorten <= 0 pt](botm_24)                                     to [out = 10, in = 180] (botm_25);

		% \draw[rounded corners = 5 pt, draw = gray] 
		% (wind_start) -- ($(wind_start) + (0.5, 1)$) -- (0, 1.3) -- (1, 1.4) -- (2.5, 1.3) --(4, 1) -- (5.75, 0.5)
		% -- (6, 0) --
		% (5.75, -0.5) -- (4, -1) -- (2.5, -1.3) -- (1, -1.4) -- (0, -1.3) -- ($(wind_start) + (0.5, -1)$) -- cycle;

		% \draw[draw = none, pattern = north east lines, pattern color = gray]
		% (wind_start) -- ($(wind_start) + (0.5, 1)$) -- (0,1.3) -- (1, 1.4) -- (2.5, 1.3) --(4, 1) -- (5.75, 0.5) -- (6, 0) -- (6, 2) -- ($(wind_start) + (0, 2)$) -- cycle;

		% \draw[draw = none, pattern = north east lines, pattern color = gray] 
		% (wind_start) -- ($(wind_start) + (0.5, -1)$) -- (0, -1.3) -- (1, -1.4) -- (2.5, -1.3) --(4, -1) -- (5.75, -0.5) -- (6, 0)-- (6, -2) -- ($(wind_start) + (0, -2)$) -- cycle;

	\end{tikzpicture}
	\caption{Hi}
\end{figure}


\begin{figure}[H]
	\centering
	\begin{tikzpicture}
		\newcount\scale
		\advance\scale by 3
		
		\newcount\angle
		\advance\angle by -30

		\draw[scale = \the\scale, rotate = \the\angle] plot file{../airfoils/NACA_0030.dat} -- cycle;

		\draw [ rotate = \the\angle] (0, 0) to coordinate[pos=0] (LE) coordinate[pos=0.5] (mid)  coordinate[pos=1] (TE) (\the\scale, 0);
		% \node [below left, rotate = \the\angle] at (mid) [yshift = 3 pt] {Chord};
		\node [below left] at (mid) [yshift = 5 pt] {$c$};

		\draw[dashed, rotate = \the\angle] (LE) -- ++(-3, 0) node (chord){};
		\draw[pointer] (LE) + (0.5, 1)     -- node[near start, above]{Leading Edge} (LE);
		\draw[pointer] (\the\scale + 1, -1) -- node[near start, above]{Trailing Edge} (TE);

		\draw[pointer, thick, shorten >= 0 pt, blue, shorten >= 2 pt] (-3, 0) to coordinate [pos = 0.99] (wind) node [near start, above]{$V_\infty$} (LE);
		\pic [draw,<-, "$\alpha$", angle eccentricity=1.5, angle radius = 20 pt] {angle = chord--LE--wind};

		% \draw[pointer, thick, shorten >= 0 pt, blue, shorten >= 2 pt, in=180] (-3, -0.3) -- ($(wind)+(0, -0.3)$);
		% \draw[pointer, thick, shorten >= 0 pt, blue, shorten >= 2 pt, in=180] (-3, -0.6) -- ($(wind)+(0, -0.6)$);


	\end{tikzpicture}
	\caption{Hi}
\end{figure}

\end{document}
