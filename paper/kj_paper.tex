\documentclass[letterpaper, twoside, 12 pt]{article}
\usepackage{james_kj}

% Use this to toggle professionalism
\newif\ifpresstime
\presstimefalse
\presstimetrue

\ifpresstime
	\renewcommand{\meo}[1]{}
\fi



\title{The Kutta--Joukowski Theorem}
\date{\today}
\author{James McFeeters}
\begin{document}
	\maketitle
	\newcounter{y}
	\setcounter{y}{0}

\begin{abstract}
	This paper will present, in full and with explanation, a derivation of the Kutta--Joukowski theorem.
	Because even if the FAA will never let me fly, they can't stop me from studying.
\end{abstract}

\section{Introduction} % (fold)
\label{sec:introduction}
	The Kutta--Joukowski theorem deals with the lift provided by two--dimensional airfoils.
	That is, we will deal with objects and fluid flows in the complex plane.

	\begin{figure}[H]
		\centering
		\begin{tikzpicture}
			\newcount\scale
			\advance\scale by 3
			
			\newcount\angle
			\advance\angle by 0

			\draw[scale = \the\scale, rotate = \the\angle, pattern = north east lines] plot file{../airfoils/NACA_0030.dat} -- cycle;

			\draw [draw = none, rotate = \the\angle] (0, 0) to coordinate[pos=0] (LE) coordinate[pos=0.5] (mid)  coordinate[pos=1] (TE) (\the\scale, 0);

			\newcount\wdist
			\advance\wdist by -2

			\newdimen\wspace
			\advance\wspace by 2 pt

			\newdimen\wgap
			\advance\wgap by 5 pt

			\draw[wind](\the\wdist, 0) to coordinate [pos = 0] (wind_start) (LE);
			\draw[wind](TE) -- ($(TE) + (-\the\wdist, 0)$);


			\coordinate(top_1) at (0.175\the\scale , 0.4\the\scale);
			% \draw (top_1) circle(2 pt);

			\coordinate(top_2) at ($(top_1) + (0.5, 0.15)$);
			% \draw (top_2) circle(2 pt);

			\coordinate(top_3) at ($(top_2) + (1, -0.075)$);
			% \draw (top_3) circle(2 pt);

			\coordinate(top_4) at ($(top_3) + (1, -0.23)$);
			% \draw (top_4) circle(2 pt);

			\coordinate(top_5) at ($(top_4) + (2, -0.09)$);
			% \draw (top_5) circle(2 pt);

			\draw[wind]($(wind_start)+(0, \the\wgap )$) to [out = 0, in = 190] (top_1);
			\draw[wind, shorten <= 0 pt](top_1) [out = 10, in = 190] to (top_2);
			\draw[wind, shorten <= 0 pt](top_2) [out = 10, in = 170] to (top_3);
			\draw[wind, shorten <= 0 pt](top_3) [out = -10, in = 170] to (top_4);
			\draw[wind, shorten <= 0 pt](top_4) [out = -10, in = 180] to (top_5);


			\coordinate(botm_1) at (0.175\the\scale , -0.4\the\scale);
			% \draw (botm_1) circle(2 pt);

			\coordinate(botm_2) at ($(botm_1) + (0.5, -0.15)$);
			% \draw (botm_2) circle(2 pt);

			\coordinate(botm_3) at ($(botm_2) + (1, 0.075)$);
			% \draw (botm_3) circle(2 pt);

			\coordinate(botm_4) at ($(botm_3) + (1, 0.23)$);
			% \draw (botm_4) circle(2 pt);

			\coordinate(botm_5) at ($(botm_4) + (2, 0.09)$);
			% \draw (botm_5) circle(2 pt);

			\draw[wind]($(wind_start)+(0, -\the\wgap )$) to [out = 0, in = 170] (botm_1);
			\draw[wind, shorten <= 0 pt](botm_1) [out = -10, in = 170] to (botm_2);
			\draw[wind, shorten <= 0 pt](botm_2) [out = -10, in = 190] to (botm_3);
			\draw[wind, shorten <= 0 pt](botm_3) [out = 10, in = 190] to (botm_4);
			\draw[wind, shorten <= 0 pt](botm_4) [out = 10, in = 180] to (botm_5);


			% \draw[pointer, shorten >= 0 pt, red, shorten >= 2 pt, in=180, out = 0] (\the\wdist, 0.6) to coordinate[pos = 0.99](a) ($(wind)+(0, 0.6)$);
			% \draw[pointer, shorten >= 0 pt, red, shorten >= 2 pt, in=180, out = 0] ($(a)+(\the\wspace, 0)$) to coordinate[pos = 0.7](ab) ($(a)+(-\the\wdist, -0.15)$);

			% \draw[pointer, shorten >= 0 pt, blue, shorten >= 2 pt, in=180] (\the\wdist, 0.3) -- ($(wind)+(0, 0.3)$);

			% \draw[pointer, shorten >= 0 pt, blue, shorten >= 2 pt, in=180] (\the\wdist, -0.3) -- ($(wind)+(0, -0.3)$);
			% \draw[pointer, shorten >= 0 pt, blue, shorten >= 2 pt, in=180] (\the\wdist, -0.6) -- ($(wind)+(0, -0.6)$);



		\end{tikzpicture}
		\caption{Hi}
	\end{figure}


	\begin{figure}[H]
		\centering
		\begin{tikzpicture}
			\newcount\scale
			\advance\scale by 3
			
			\newcount\angle
			\advance\angle by -30

			\draw[scale = \the\scale, rotate = \the\angle] plot file{../airfoils/NACA_0030.dat} -- cycle;

			\draw [ rotate = \the\angle] (0, 0) to coordinate[pos=0] (LE) coordinate[pos=0.5] (mid)  coordinate[pos=1] (TE) (\the\scale, 0);
			% \node [below left, rotate = \the\angle] at (mid) [yshift = 3 pt] {Chord};
			\node [below left] at (mid) [yshift = 5 pt] {$c$};

			\draw[dashed, rotate = \the\angle] (LE) -- ++(-3, 0) node (chord){};
			\draw[pointer] (LE) + (0.5, 1)     -- node[near start, above]{Leading Edge} (LE);
			\draw[pointer] (\the\scale + 1, -1) -- node[near start, above]{Trailing Edge} (TE);

			\draw[pointer, thick, shorten >= 0 pt, blue, shorten >= 2 pt] (-3, 0) to coordinate [pos = 0.99] (wind) node [near start, above]{$V_\infty$} (LE);
			\pic [draw,<-, "$\alpha$", angle eccentricity=1.5, angle radius = 20 pt] {angle = chord--LE--wind};

			% \draw[pointer, thick, shorten >= 0 pt, blue, shorten >= 2 pt, in=180] (-3, -0.3) -- ($(wind)+(0, -0.3)$);
			% \draw[pointer, thick, shorten >= 0 pt, blue, shorten >= 2 pt, in=180] (-3, -0.6) -- ($(wind)+(0, -0.6)$);


		\end{tikzpicture}
		\caption{Hi}
	\end{figure}
% section introduction (end)

\clearpage
\nocite{*}
\bibliographystyle{plainnat}
\bibliography{kj_references}

\end{document}
